\documentclass{article}
\usepackage[UTF8]{ctex}
\usepackage{graphicx} 
\usepackage{pythonhighlight,listings}
\usepackage{amsmath}
\usepackage{amsfonts,amssymb}
\begin{document}
\section{方法}
\subsection{环境设计}
%这部分说法可以从化学角度更专业一点?
我们将卡宾分子的构建过程建模为一个有限步长的序列决策问题。构建单元(blocks)分为两类:\textit{core} 与 \textit{substructure}。候选集合包含 30 个 core 和 852 个 substructure。初始状态 $s_0$ 为空图。智能体首先必须选择且仅选择一个 core,随后在其反应位点上执行结构扩展。

在任意状态 $s_t$,动作空间由两类操作组成:\textit{add} 与 \textit{combine}。其中,\textit{add} 表示在选定位点接入一个 substructure;新接入片段可引入额外可反应位点,从而支持后续扩展。\textit{combine} 表示连接两个已有可反应位点(可来自 core 或不同 substructure),用于形成更紧凑的拓扑结构。最终,包含单一 core 且任意数量 substructure 的分子作为有效终止状态。
\subsection{代理模型}
%代理模型的必要性
由于真实评估函数(oracle)的评估成本非常高,我们使用代理模型来评估候选分子的性质。代理模型我们使用GraphGPS~\cite{graphgps},该模型是一个基于图神经网络的代理模型,能够预测分子的性质。


\subsection{多目标生成式模型}
我们采用多目标生成范式,将性质优化表示为向量奖励 $\mathbf{r}(x)=[r_1(x),\dots,r_m(x)]$,并通过条件化偏好向量 $\mathbf{w}$ 学习一族策略 $\pi_\theta(\cdot|s,\mathbf{w})$。该设计使模型能够在单次训练后覆盖不同目标权衡,并在采样时按需求探索 Pareto 区域。实现上,我们参考 GFlowNet 的流匹配思想以提升对高奖励区域的覆盖能力~\cite{bengio2021flow}。
\subsection{主动学习逻辑}
我们采用“生成--评估--回流训练”的主动学习闭环。每轮迭代中,模型先生成候选分子并进行性质评估;随后根据不确定性与非支配排序联合选择样本,优先保留位于或接近 Pareto front 的高信息样本;最后将新增标注样本并入训练集更新策略模型。该流程在控制评估成本的同时,持续提升模型在 Pareto 前沿附近的采样效率与解的质量。
\bibliography{reference}
\bibliographystyle{IEEEtran}
\end{document}