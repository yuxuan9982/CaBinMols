\documentclass{article}
\usepackage[UTF8]{ctex}
\usepackage{graphicx} 
\usepackage{pythonhighlight,listings}
\usepackage{amsmath}
\usepackage{amsfonts,amssymb}
\begin{document}
\section{方法}
\subsection{环境设计}
我们将卡宾分子的构建过程建模为一个有限步长的序列决策问题。构建单元(blocks)分为两类:\textit{core} 与 \textit{substructure}。候选集合包含 30 个 core 和 852 个 substructure。初始状态 $s_0$ 为空图。智能体首先必须选择且仅选择一个 core,随后在其反应位点上执行结构扩展。

在任意状态 $s_t$,动作空间由两类操作组成:\textit{add} 与 \textit{combine}。其中,\textit{add} 表示在选定位点接入一个 substructure;新接入片段可引入额外可反应位点,从而支持后续扩展。\textit{combine} 表示连接两个已有可反应位点(可来自 core 或不同 substructure),用于形成更紧凑的拓扑结构。最终,包含单一 core 且任意数量 substructure 的分子作为有效终止状态。

由于真实评估函数(oracle)的计算代价较高,我们以代理模型 $\hat{f}_{\phi}$ 近似真实性质映射 $f$,并将其用于候选分子的快速筛选。对任意分子图 $G=(V,E)$,代理模型的输入由两部分组成:其一是原子/键构成的拓扑与化学特征;其二是图级全局标量 $u=dE\_{triplet}$。模型首先通过图编码器得到结构表示 $\mathbf{h}_G=\mathrm{Enc}_{\phi}(G)$,再将 $\mathbf{h}_G$ 与 $u$ 融合并输入回归层,输出三维性质预测向量
\[
\hat{\mathbf{y}}=\hat{f}_{\phi}(G,u)=\big[\hat{y}_1,\hat{y}_2,\hat{y}_3\big]
=\big[\widehat{dE\_{triplet}},\widehat{vbur\_ratio\_vbur\_vtot},\widehat{dE\_{AuCl}}\big].
\]
训练阶段采用多目标联合回归,记真实标签为 $\mathbf{y}=[y_1,y_2,y_3]$,则优化目标写为
\[
\mathcal{L}_{proxy}=\sum_{k=1}^{3}\lambda_k\,\ell\!\left(\hat{y}_k,y_k\right),
\]
其中 $\ell(\cdot,\cdot)$ 为回归损失(实验中使用 MAE),$\lambda_k$ 为各性质权重。该设计的核心动机是:$dE\_{triplet}$ 不仅是预测目标,同时也可作为描述电子结构状态的先验信号,与其余目标存在耦合关系;将其显式注入图级表示后,模型可在共享表示空间中学习跨目标相关性,从而提高多目标预测的稳定性与样本效率。推理时,代理模型输出 $\hat{\mathbf{y}}$ 作为后续多目标打分与 Pareto 筛选的依据。


\subsection{GFlowNet}
传统强化学习(RL)通常以“给定目标下求解单一最优策略”为核心,即倾向于生成奖励最高的一条动作序列。近年来研究表明,在许多实际任务中,“生成一组具有多样性的高质量候选解”往往比“仅输出一个全局最优解”更有价值,这一点在分子设计与强化学习探索中尤为明显。以分子设计为例,模型不应只给出一个分数最高但难以合成的分子,而应提供一批性能接近最优、但在可合成性等维度上更具可操作性的候选分子,以支持后续实验筛选与决策。

GFlowNet 的目标不是学习单一路径最优,而是学习一个在终止状态集合 $\mathcal{X}$ 上的采样分布 $p_\theta(x)$,使其与奖励函数成正比:
\[
p_\theta(x)\propto R(x),\quad x\in\mathcal{X},\;R(x)>0.
\]
在状态转移图中,设 $F_\theta(s\!\rightarrow\!s')$ 为边流、$F_\theta(s)$ 为状态总流,则对任一非初始且非终止中间状态需满足流守恒:
\[
\sum_{s''\rightarrow s}F_\theta(s''\!\rightarrow\!s)=\sum_{s\rightarrow s'}F_\theta(s\!\rightarrow\!s').
\]
对终止状态 $x$,其入流等于奖励,即 $F_\theta(x)=R(x)$。在本文场景中,终止状态对应完整分子;我们将代理模型给出的多性质评分聚合为标量奖励并做正值化处理,以保证可用于流匹配训练。

实现上,我们采用前向策略 $P_F(a_t|s_t)$ 逐步执行 \textit{add}/\textit{combine}/\textit{stop} 动作生成分子,并以反向策略 $P_B(s_t|s_{t+1})$ 近似逆过程。给定轨迹 $\tau=(s_0\!\rightarrow\!s_1\!\rightarrow\!\cdots\!\rightarrow\!x)$,训练时使用 Trajectory Balance 目标:
\[
\mathcal{L}_{TB}=
\left(
\log Z_\theta+\sum_{t=0}^{T-1}\log P_F(s_{t+1}|s_t)
-\log R(x)-\sum_{t=0}^{T-1}\log P_B(s_t|s_{t+1})
\right)^2,
\]
其中 $Z_\theta$ 为可学习配分函数。该目标直接约束整条生成轨迹的前向概率与终止奖励一致,从而在高奖励区域保持采样强度的同时避免策略塌缩到单一结构,提高候选分子的多样性与可探索性。

\subsection{多目标GFlowNet}
单目标 GFlowNet 仅对应一个标量奖励 $R(x)$,而在本任务中每个分子同时对应多维性质向量 $\mathbf{r}(x)=[r_1(x),\dots,r_m(x)]$。为在一次训练中覆盖不同目标权衡,我们引入偏好向量 $\boldsymbol{\omega}\in\Delta^{m-1}$,并学习条件化策略
\[
\pi_\theta(\cdot|s,\boldsymbol{\omega}),\quad P_F(\cdot|s,\boldsymbol{\omega}),\quad P_B(\cdot|s',\boldsymbol{\omega}).
\]
其中,训练时偏好并非固定,而是从分布 $p(\boldsymbol{\omega})$ 采样;$p(\boldsymbol{\omega})$ 将直接影响模型覆盖的 Pareto 前沿区域。本文采用
\[
\boldsymbol{\omega}\sim \mathrm{Dirichlet}(\boldsymbol{\alpha}),
\]
并在输入策略网络时对 $\boldsymbol{\omega}$ 使用 thermometer encoding(离散分桶的单调累计编码)以增强偏好条件信号表达。给定 $\boldsymbol{\omega}$ 后,将向量奖励标量化为
\[
R(x|\boldsymbol{\omega})=g\!\left(\boldsymbol{\omega},\mathbf{r}(x)\right),
\]
并引入奖励指数 $\beta>0$,使目标分布满足
\[
\pi(x|\boldsymbol{\omega})\propto R(x|\boldsymbol{\omega})^{\beta}.
\]
该设计会强化策略对 $R(x|\boldsymbol{\omega})$ 模式区域(高奖励峰值区域)的关注,从而更易生成高质量且保持多样性的候选分子。结合 TB 训练目标,可写为
\[
\mathcal{L}_{MTB}=
\left(
\log Z_\theta(\boldsymbol{\omega})
+\sum_{t=0}^{T-1}\log P_F(s_{t+1}|s_t,\boldsymbol{\omega})
-\log R(x|\boldsymbol{\omega})
-\sum_{t=0}^{T-1}\log P_B(s_t|s_{t+1},\boldsymbol{\omega})
\right)^2.
\]
推理阶段通过采样 $\boldsymbol{\omega}$,可获得覆盖 Pareto 前沿不同区域的一组候选分子;不同 $p(\boldsymbol{\omega})$ 的影响在实验部分进行对比分析。

\subsection{主动学习}
我们采用“生成--评估--回流训练”的主动学习闭环。首先,使用初始标注数据集 $D_0$ 训练代理模型与生成策略。随后在第 $k$ 轮迭代中,模型先生成一批候选分子并由代理模型进行快速打分;再结合预测不确定性与非支配排序(Pareto ranking)选择高信息样本,优先保留位于或接近 Pareto front 的候选;最后对入选样本进行真实评估,并将新获得的标注数据并入训练集 $D_{k+1}=D_k\cup\mathcal{B}_k$,用于下一轮模型更新。该流程在控制真实评估成本的同时,持续提升模型在 Pareto 前沿附近的采样效率与候选质量。

\subsection{离线+在线,模型迭代最可靠的方式}
随着主动学习迭代推进,会累积大量离线数据(包含代理预测样本与真实评估样本)。在大规模、稀疏高分区域的分子搜索任务中,这些数据不仅能提升样本效率,也能显著稳定训练过程。基于此,我们采用“离线+在线”联合迭代范式:离线阶段利用历史数据学习稳定分布,在线阶段依托代理奖励持续探索新区域,实现“利用--探索”的动态平衡。

具体而言,离线部分采用 COFlowNet 框架,通过区分受支持与不受支持转移,缓解纯离线训练中目标分布偏移问题;在线部分采用 GFlowNet-proxy,在离线覆盖不足区域提供更强的外推能力。我们进一步提出不确定性驱动的混合前向策略。状态 $s$ 的不确定性由深度集成(Deep Ensemble)代理模型估计为 $u(s)$,并将其映射为混合系数
\[
\alpha_1(s)=\exp\!\left(-\frac{u(s)}{\tau}\right),\qquad
\alpha_2(s)=1-\alpha_1(s),
\]
其中 $\tau>0$ 为温度参数。最终前向策略写为
\[
P_F(\cdot|s)=\alpha_1(s)\,P_F^{\mathrm{COF}}(\cdot|s)+\alpha_2(s)\,P_F^{\mathrm{proxy}}(\cdot|s).
\]
当 $u(s)$ 较小(离线覆盖充分)时,策略更依赖 COFlowNet;当 $u(s)$ 较大(离线覆盖稀疏)时,策略自动提高对在线策略的权重,从而在保证稳定性的同时增强对新颖高价值分子的发现能力。除上述策略外,我们还设计了其他几种混合策略,包括:
\begin{itemize}
\item \textbf{阈值切换策略(Hard Switch)}:设不确定性阈值 $\delta$,
\[
P_F(\cdot|s)=
\begin{cases}
P_F^{\mathrm{COF}}(\cdot|s), & u(s)\le \delta,\\
P_F^{\mathrm{proxy}}(\cdot|s), & u(s)>\delta.
\end{cases}
\]
该策略具有明确的决策边界,但可能在阈值附近产生不连续切换。
\item \textbf{竞争式 Softmax 融合(Score-based Softmax)}:构造两路置信分数 $q_1(s),q_2(s)$,并令
\[
\alpha_i(s)=\frac{\exp(q_i(s)/\tau)}{\sum_{j=1}^{2}\exp(q_j(s)/\tau)},\quad
P_F(\cdot|s)=\sum_{i=1}^{2}\alpha_i(s)\,P_F^{(i)}(\cdot|s),
\]
其中 $P_F^{(1)}=P_F^{\mathrm{COF}}$,$P_F^{(2)}=P_F^{\mathrm{proxy}}$。该策略平滑且可学习。
\item \textbf{阶段性退火融合(Iteration Annealing)}:在第 $k$ 轮主动学习时设置
\[
\alpha_1^{(k)}=\max\!\left(\alpha_{\min},1-\frac{k}{K}\right),\qquad
\alpha_2^{(k)}=1-\alpha_1^{(k)},
\]
\[
P_F^{(k)}(\cdot|s)=\alpha_1^{(k)}P_F^{\mathrm{COF}}(\cdot|s)+\alpha_2^{(k)}P_F^{\mathrm{proxy}}(\cdot|s).
\]
该策略前期偏向离线稳定训练,后期逐步增强在线探索能力。
\end{itemize}

由此,我们提出了 Offline with Online Multi-Objective GFlowNet(OWOMGFN)。其核心思想是统一
COFlowNet 的离线稳定性与 GFlowNet-proxy 的在线探索能力:先利用历史数据进行离线流匹配,在受支持转移上学习稳健策略;再在每轮主动学习中基于代理模型对新候选进行快速评估,并将高价值样本回流至训练集持续迭代。具体而言,我们首先将 COFlowNet 扩展到多目标场景。由于离线数据可提供准确的多目标奖励,训练时可通过采样不同偏好向量 $\boldsymbol{\omega}$(或等价地调节 Dirichlet 参数 $\boldsymbol{\alpha}$)学习对 Pareto 前沿不同区域的覆盖。随后,我们引入不确定性驱动的混合前向策略,在“已知高置信区域”偏向离线策略,在“未知稀疏区域”自动提升在线策略权重,从而在样本效率、稳定性与新颖性之间取得更优平衡。考虑到本任务中 oracle 评估耗时显著高于 GFlowNet 训练与采样开销,我们在每轮主动学习迭代中均基于最新数据重新训练离线策略、在线策略与代理模型。

具体的训练与采样流程如算法~\ref{alg:owomgfn} 所示。

\begin{lstlisting}[language=Python,caption={OWOMGFN training and sampling pipeline},label={alg:owomgfn}]
# Input:
# D0: initial labeled dataset
# K : active learning rounds
# tau: uncertainty temperature
# alpha: Dirichlet parameter for preference sampling
#
# Models:
# PF_COF, PB_COF      : offline forward/backward policies (COFlowNet)
# PF_proxy, PB_proxy  : online forward/backward policies (GFlowNet-proxy)
# f_hat_ens           : deep-ensemble proxy for reward + uncertainty

D = D0

for k in range(1, K + 1):
    # Stage 1: retrain all models on the latest dataset
    Train(f_hat_ens, on=D, objective="multi-objective regression")
    Train(PF_COF, PB_COF, on=D,
          objective="offline multi-objective TB/flow matching",
          pref_sampler=Dirichlet(alpha))
    Train(PF_proxy, PB_proxy, on=D, reward=f_hat_ens,
          objective="online multi-objective TB",
          pref_sampler=Dirichlet(alpha))

    # Stage 2: hybrid policy rollout
    C = []  # generated candidates
    for episode in range(N_rollout):
        omega = SampleDirichlet(alpha)
        s = s0
        while not terminal(s):
            u = Uncertainty(f_hat_ens, s)
            alpha1 = exp(-u / tau)
            alpha2 = 1 - alpha1
            PF_mix = alpha1 * PF_COF(. | s, omega) + alpha2 * PF_proxy(. | s, omega)
            a = SampleAction(PF_mix)
            s = Transition(s, a)
        C.append(s)  # terminal molecule

    # Stage 3: proxy evaluation + Pareto selection
    Y_hat = PredictObjectives(f_hat_ens, C)  # multi-objective prediction
    U_hat = PredictUncertainty(f_hat_ens, C)
    B = SelectByParetoAndUncertainty(C, Y_hat, U_hat, budget=B_k)

    # Stage 4: expensive oracle labeling and dataset update
    Y_true = OracleEvaluate(B)
    D = D union {(x, y) for x, y in zip(B, Y_true)}

return Trained policies and Pareto-diverse molecule set
\end{lstlisting}

\bibliography{reference}
\bibliographystyle{IEEEtran}
\end{document}